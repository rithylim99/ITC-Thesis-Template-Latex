\section*{\centering \KhOSml សេចក្តីសង្ខេប}
\addcontentsline{toc}{section}{\KhOSml សេចក្តីសង្ខេប}
{\KhOS{និក្ខេបបទនេះត្រូវបានសរសេរឡើងក្នុងគោលបំណងសិក្សាលើម៉ូដែលគណិតវិទ្យា និង ប្រ
ព័ន្ធបញ្ជារបស់យន្តហោះគ្មានមនុស្សបើកមួយប្រភេទហៅថាដ្រូនដើម្បីជាជំនួយដល់ការអភិវឌ្ឍន៍បច្ចេកវិទ្យាដ្រូនក្នុងការសិក្សាស្រាវជ្រាវ។ ការសិក្សានេះផ្តោតសំខាន់ទៅលើចំនុចធំៗចំនួនបី ដែលទី១សិក្សាអំពីការធ្វើម៉ូដែលរបស់ដ្រូន ទី២សិក្សាទៅលើការបង្កើតប្រព័ន្ធបញ្ជាសាមញ្ញមួយដើម្បីអោយដ្រូនមានលំនឹងនៅលើលំហរអាកាស និង ចំនុចចុងក្រោយគឺ ការធ្វើតេស្តម៉ូដែលគណិតវិទ្យាក្នុងកុំព្យូទ័រដោយប្រើកម្មវិធីជំនួយ (Simulation) មុននឹងធ្វើពិសោធន៍ទៅលើដ្រូនជាក់ស្តែង។
ម៉ូដែលគណិតវិទ្យានេះត្រូវបានបង្កើតឡើងដោយការសិក្សាស៊ីជម្រៅទៅលើឌីណាមិចរបស់ប្រព័ន្ធដ្រូន។ បន្ទាប់មក គេធ្វើពិសោធន៍មួយចំនួនដើម្បីទាញយកប៉ារ៉ាម៉ែត្រសំខាន់ៗសម្រាប់គណនារកសមីការឌីណាមិចខាងលើ។ ក្រោយមកយើងធ្វើ Simulation ដោយប្រើប្រាស់ “Matlab/
Simulink Blockset”។ ក្រោយមកទៀត ប្រព័ន្ធបញ្ជា PID មួយក្រុមត្រូវបានរចនាឡើងក្នុងគោលបំណងបញ្ជាមុំរបស់ដ្រូនឲ្យមានលំនឹង។ ប្រភេទបញ្ជាដែលបានស្នើឡើងមានឈ្មោះ"ប្រភេទបញ្ជា Acrobatic"។ ការសិក្សានេះមិនបានផ្តោតទៅលើការបញ្ជាទីតាំង និងស៊ីនេម៉ាទិចរបស់ប្រព័ន្ធដ្រូនឡើយ។ ជាចុងក្រោយ ម៉ូដែលគណិតវិទ្យា និងប្រព័ន្ធបញ្ជាត្រូវបានតេស្តដោយប្រើប្រាស់កម្មវិធី Matlab/Simulink ហើយប្រព័ន្ធបញ្ជាដែលបានតេស្តរួចត្រូវបានយកទៅធ្វើពិសោធន៍លើដ្រូនជាក់ស្តែងដើម្បីផ្ទៀងផ្ទាត់លទ្ធផល។ លទ្ធផលដែលទទួលបាន បានបញ្ជាក់ឲ្យឃើញថា ដ្រូនត្រូវបានបញ្ជាប្រកបដោយភាពសុក្រិត្យតាមវិធីសាស្ត្រដែលបានលើកមកសិក្សាខាងលើ។
}
}